\newpage
\section{Compact Operators}
\(X, Y\) Banach spaces. 
\begin{definition}
    A linear operator \(K:X \to Y\) is said to be compact if \(\forall \; E \subseteq X\) bounded, the set \(K(E)\) is relatively compact, namely \(\overline{K(E)}\) is compact. 

    Equivalently, \(K\) is compact if \(\forall \; \{x_n\} \subset X \) bounded, the sequence \(\{K(x_n)\}\) has a strongly convergent subsequence. 
\end{definition}
\begin{proposition}
    \(K:X \to Y\) linear and compact. Then \(K \in \mathcal{L}(X, Y)\)
\end{proposition}
\begin{proof}
    Define \(B:=\overline{B_1(0)}\) in \(X\). If \(K\) is compact 
    \(\Rightarrow K(B) \) is relatively compact 
    \(\Rightarrow \overline{K(B)}\) is compact 
    \(\Rightarrow \overline{K(B)}\) is bounded
    \(\Rightarrow K(B)\) is bounded: \(\exists \; M>0 \) s.t. 
    \[
        \norm{Kx}_Y \leq M \qquad \forall \; x \in \overline{B_1(0)} = B
    \]
    \[
        \Rightarrow \stackbelow{\underbrace{\sup_{\norm{x} \leq 1} \norm{Kx}_Y}}{\norm{K}_{\mathcal{L}(X, Y)}} \leq M
    \]
\end{proof}

\begin{definition}
    \(T \in \mathcal{L}(X, Y)\) has finite rank if 
    \[
        \underset{\dim (T(X))}{\text{ the image of }T= \{y \in Y: y = Tx\} \text{ for some } x \in X} < \infty
    \]
\end{definition}
\(T(X) \subset Y\) is a subspace.
\begin{proposition}
    \(T \in \mathcal{L}(X, Y)\) has finite rank \(\Rightarrow T\) is compact.
\end{proposition}
\begin{proof}
    \(A \subset X \) bounded. Since \(T \in \mathcal{L}(X, Y)\), then \(T(A)\) is bounded. \(T(A) \subset T(X) \approx \real^n\), since \(T\) has finite rank. 
    
    Thus, \(T(A)\) is a bounded set of \(\real^n\) \(\Rightarrow T(A) \) is relatively compact. 
\end{proof}

\begin{definition}
    We denote by \(\mathcal{K}(X, Y)\) the class of linear compact operators from \(X \) to \(Y\). This is a linear subspace.

    If \(Y=X\), we write \(\mathcal{K}(X)\)
\end{definition}
\begin{proposition}
    \(X, Y\) Banach spaces, \(T:X \to Y\) linear and compact, \(Y\) in \(\infty\) dim. Then \(T\) cannot be surjective.
\end{proposition}
\begin{proof}
    Recall that \(C\) compact set, \(S \subset C\) closed \(\Rightarrow S\) is compact (in any metric space)

    Assume by contradiction that \(T\) is surjective. By the OMT, \(T\) is an open map. Take 
    \[
        \emptyset \neq A \subset X
    \]
    open and bounded. \(T(A)\) is relatively compact (since \(T\) is compact), and is open (since \(T\) is an open map) 
    and \(\neq \emptyset\)
    \[
        \Rightarrow T(A) \supset B_r(y_0)
    \]
    for some \(y_0 \in Y \) and \(r>0\). Thus, 
    \[
        \overline{T(A)} \supset \overline{B_r(y_0)} \Rightarrow \overline{B_r(y_0)} \text{ is compact in } Y.
    \] 
    This contradicts the Riesz theorem, since in \(\infty\) dimension balls are never compact. 
\end{proof}

\begin{proposition}
    \(X\), \(Y\), \(Z\) Banach spaces. \(T \in \mathcal{L}(X, Y)\), \(S \in \mathcal{K}(Y, Z)\) (or \(T \in \mathcal{K}(X, Y)\), \(S \in \mathcal{L}(Y, Z)\)). Then \(S \circ T\) is compact.
\end{proposition}
\begin{theorem}
    \(\mathcal{K}(X, Y) \) is a closed subspace of \(\mathcal{L}(X, Y)\). \(\Rightarrow (\mathcal{K}(X, Y), \normdot_{\mathcal{K}(X, Y)})\) is a Banach space.
\end{theorem}
Consequence: if we want to check that \(T \in \mathcal{L}(X, Y)\) is compact, we can prove that \(\exists\; \{T_n\} \subseteq \mathcal{K}(X, Y)\) s.t. 
\[
    \norm{T_n - T}_{\mathcal{L}} \to 0
\]
Since \(\mathcal{K}(X, Y)\), it follows that \(T\) is compact.