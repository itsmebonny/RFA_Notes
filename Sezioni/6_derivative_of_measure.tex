\newpage
\section{Derivative of a measure}
\subsection{Radon-Nikodym derivative}
\(\left(X, \mathcal{M}, \mu \right)\) measure space,
\(\oldphi : X \to \left[0, \infty \right]\) measurable.  

We learned that \(\nu: \mathcal{M} \to \left[0, \infty \right]\) by 
\[
    \nu(E)= \int_E \oldphi \, d\mu \mbox{ is a measure on }(X, \mathcal{M})
\] 

If the equation above holds, then we say that \(\oldphi\) is the \textbf{Radon Nikodym derivative} of \(\nu\) with respect to \(\mu\), and we write 
\[
    \oldphi = \frac{d\nu}{d\mu}
\]
\begin{definition}
    \(\mu, \nu  \) measures on \(\left(X, \mathcal{M}\right)\). 
    We say that \(\nu\) is absolutely continuous with respect to \(\mu\), \(\nu \ll \mu \) if 
    \[
        \mu(E) = 0 \Rightarrow \nu(E)=0
    \]
\end{definition}

\begin{lemma}
    There is a necessary condition:
    \[ 
        \exists \; \frac{d \nu}{d \mu} \Rightarrow \nu \ll \mu 
    \]
\end{lemma}

\begin{proof}
    \[
        \nu(E) = \int_E \left(\frac{d\nu}{d\mu}\right) \, d\mu = 0
    \] 
    if \(\mu(E)=0\) by basic properties of \(\int\)
\end{proof}
\subsection{Radon-Nikodym theorem}
\begin{theorem}[Radon Nikodym Theorem]
    \(\left(X, \mathcal{M}\right) \) measurable space, \(\mu, \nu\) measures. \\
    If \(\nu \ll \mu \) and moreover \(\mu \) is \(\sigma\)-finite, then \(\oldphi : X \to \left[0, \infty\right]\) measurable s.t.
    \[
        \oldphi = \frac{d \nu}{d \mu} \quad  \text{ namely } \nu(E)= \int_E \oldphi \, d\mu \quad \forall \; E \in \mathcal{M}
    \]
\end{theorem}

\begin{remark}
    If \(\mu\) is not sigma finite the theorem may fail.
\end{remark}
\begin{example}
    In \(\left(\left[0, 1\right], \mathcal{L}\left(\left[0, 1\right]\right)\right)\) consider the counting measure \(\mu = \mu_C\) and the Lebesgue measure \(\nu= \lambda\)
    \(\nu \ll \mu\) since \(\mu(E)= 0 \Leftrightarrow E= \emptyset \Rightarrow \lambda(E) = \nu(E)=0\) \\
    But we can check that \( \nexists \; \oldphi : \left[0, 1\right] \rightarrow \left[0, \infty \right]\) measurable s.t. \(\lambda(E)= \int_E \oldphi \, d\mu_C\)

    Check by contradiction: assume that \(\oldphi \) does exist, and take \(x_0 \in \left[0, 1\right]\)
    \[ 
        0 = \lambda (\left\{x_0\right\}) = \int_{\left\{x_0\right\}} \oldphi \, d\mu_C = \oldphi (x_0) \; \overbrace{\mu_C (\left\{x_0\right\})}^{=1}= \oldphi (x_0) \Rightarrow \oldphi (x_0) = 0 \; \forall \; x_0 \in \left[0, 1\right]\]
    
    But then \(1 = \lambda(\left[0, 1\right]) = \int_{\left[0, 1\right]} 0 \, d\mu_C = 0\). Contradiction.
    Note that \(\mu_C (\left[0, 1\right]) = \infty \) and \(\left( \left[0,1\right], \mathcal{L}(\left[0, 1\right]), \mu_C\right)\) is not \(\sigma\)-finite (\(\left[0,1\right]\) is uncountable)
\end{example}
