\newpage
\section{Weak Convergence}
We know that \(\overline{B_1(0)}\) is never compact in \(\infty\) dimension. This is a problem is in proving convergence of sequences. A way to approach this issue consists in weakening the notion of convergence.
\subsection{Weak convergence in Banach spaces}
\begin{definition}
    \(X\) Banach space. \(\{x_n\} \subset X\) sequence, \(x \in X\). We say that \(x_n\) tends to \(x\) \textbf{weakly} (in \(X\)) as \(n \to \infty\), \(x_n \rightharpoonup x\) in \(X\), if 
    \[
        L x_n \to Lx \qquad \forall \; L \in X^*
    \]
\end{definition}
\begin{remark}
    Assume that \(x_n \to x\) in \(X\), namely \(\norm{x_n -x}_X \to 0\). If \(f:X \to \real\) is continuous, then
    \[
        \lim_{n \to \infty} f(x_n) = f(x)
    \]
    In particular, this is true if \(f = L \in X^*\):
    \[
        \begin{array}{rlc}
            x_n \to x \in X \Rightarrow & L x_n \to Lx & \forall\; L \in X^* \\
            & x_n \rightharpoonup x \in X & \\
            x_n \to x \in X \Rightarrow & x_n \rightharpoonup x \text{ weakly } \in X & \\
             \nLeftarrow & 
        \end{array}
    \]
\end{remark}
\begin{remark}
    We will be interested in weak convergence in \(L^p\). 
    
    \noindent If \(p \in [1, \infty)\), then 
    \[
        f_n \rightharpoonup f \text{ weakly in } L^p(X) \Leftrightarrow \int_X f_n g \, d\mu \to \int_X fg \, d\mu \quad \forall \; g \in L^{p'}
    \]
    \noindent Similarly, in \(\ell^p = L^P(\natural, \mathcal{P}(\natural), \mu_c)\)
    \[
        x_n \rightharpoonup x \text{ weakly in } \ell^p \Leftrightarrow \sum_{k=1}^\infty x_n^{(k)} y^{(k)} \to \sum_{k=1}^\infty x^{(k)} y^{(k)} \quad \forall \; y \in l^{p'}
    \]
\end{remark}

\begin{proposition}
    The weak limit is unique (if it exists)
\end{proposition}
\begin{proof}
    By contradiction, suppose that \(\exists \{ x_n\} \subset X\) s.t. \(x_n \rightharpoonup x_1\), \(x_n \rightharpoonup x_2\) weakly in \(X\), \(x_1 \neq x_2\). Then
    \[
        \begin{array}{lc}
            Lx_n \to Lx_1 & \forall\; L \in X^* \\
            Lx_n \to L x_2 & \forall\; L \in X^* \\
            \Rightarrow Lx_1 = Lx_2 & \forall\; L \in X^*
    \end{array} 
    \] 
    By Hahn Banach (corollary 2), this implies \(x_1 = x_2\), a contradiction.
\end{proof}

\begin{proposition}
    If \(x_n \rightharpoonup x\) weakly in \(X\), then \(\{x_n\}\) is bounded, and 
    \[
        \norm{x} \leq \liminf_{n \to \infty} \norm{x_n} \tag*{weak lower semi continuity of \(\normdot\)}
    \]
\end{proposition}
\begin{proof}
    \begin{itemize}
        \item \(\{x_n\}\) is bounded
        
        \(x_n \rightharpoonup x\) weakly \(\Rightarrow \{L x_n\}\) is bounded in \(\real\), \(\forall \; L \in X^*\). Consider \(\Lambda_n \in X^{**}\) def by 
        \[
            \Lambda_n L = L x_n \quad \forall\; L \in X^{*}
        \]
        \(\forall\; L \in X^* \exists\; M_L >0 \) s.t.
        \[
            \abs{\Lambda_n L} = \abs{L x_n } \leq M_L \qquad \forall\; n \tag*{PB}
        \]
        (point wise boundedness of \(\{\Lambda_n \} \subset X^{**}\))
        \[
            \Lambda_n: X^{*}= \mathcal{L}(X, \real) \to \real    
        \]
        By Banach Steinhaus, \(\{\Lambda_n\}\) is uniformly bounded:
        \[
            \sup_n \norm{ \Lambda_n}_{\mathcal{L}(X^*, \real)} \leq M
        \]
        Moreover, by Hahn Banach, \(\forall \; n \in \natural \quad \exists L_n \in X^*\) s.t. \(\norm{L_n}_* = 1 \) and \(L_n x_n = \norm{x_n}\). Therefore,
        \[
            \norm{x_n} = \abs{L_n x_n} = \abs{\Lambda_n L_n} \leq \norm{\Lambda_n}_\mathcal{L} \norm{L_n}_* \leq M \quad \forall n \in \natural
        \] 
        \item \(x_n \rightharpoonup x\) weakly. 
        
        By corollary 1 of Hahn Banach, \(\exists L_x \in x^*\) s.t. \(\norm{L_x}_* = 1\) and \(L_x x = \norm{x}\). Then
        \[
            \norm{x} = \abs{L_x x} = \lim_n \abs{L_x x_n} = \liminf_n \abs{L_x x_n} \leq \liminf_n \norm{L_x}_* \norm{x_n }_X = \liminf_n \norm{x_n }_X
        \]    
    \end{itemize}
\end{proof}

\begin{proposition}
    \(x_n \rightharpoonup x \) in \(X\) weakly, and \(L_n \to L\) (strongly) in \(X^*\). Then
    \[
        L_n x_n \to Lx \quad \text{ in } \real
    \]
\end{proposition}
\begin{proposition}
    \(X, Y\) Banach, \(T \in \mathcal{L}(X, Y)\)
    \[
        x_n \rightharpoonup x \text{ weakly } \Rightarrow Tx_n \rightharpoonup Tx \text{ weakly }
    \]
\end{proposition}

We introduced the weak convergence. \(X\) Banach space. \({x_n} \subset X\) converges weakly to \(x\), \(x_n \rightharpoonup x\) weakly in \(X\), if 
\[
    L x_n \to Lx \text { in } \real, \quad \forall\; L \in X^* = \mathcal{L}(X, \real)
\]

Recall that: 
\begin{itemize}
    \item \(x_n \to x \) strongly in \(X\), namely \(\norm{x_n - x}_X \to 0\) \(\Rightarrow x_n \rightharpoonup x\) and \(\nLeftarrow\)
    \item \(x_n \rightharpoonup x \Rightarrow \{x_n\}\) is bounded, the weak limit \(x\) is unique, and 
    \[
        \norm{x} \leq \liminf _{n \to \infty} \norm{x_n}
    \]
\end{itemize}

\begin{remark}
    In \(\real^n\) (or any finite dimensional Banach space) \(x_n \rightharpoonup x\) weakly \(\Leftrightarrow x_n \to x\) strongly (ex.)
\end{remark}

With the same philosophy we introduce: 
\begin{definition}
    \(X \) Banach \(\Rightarrow X^* \) is Banach as well.
\end{definition}
\subsection{Weak\texorpdfstring{\({}^*\)}{*} convergence}
\begin{definition}
    A sequence \(\{L_n \} \subset X^*\) is \textbf{weakly\(^*\)} convergent to \(L \in  X^*\), namely \(L_n {\rightharpoonup}^* L\) in \(X^*\), if 
    \[
        L_n x \to Lx \in \real \quad \forall\; x \in X
    \]
\end{definition}
\begin{remark}
    Observe that a sequence \(\{L_n\} \) tends weakly to \(L\) in \(X^*\) if 
    \[
        \Lambda L_n \to \Lambda L \quad \forall \Lambda \in X^{**}
    \]
\end{remark}
We know that \(\exists \; \tau:X \to X^{**}\) canonical map s.t. 
\[
    \underset{X^{**}}{\langle}\tau(x), L \underset{X^*}{\rangle} = Lx \quad \forall \; L \in X^*
\]
Thus \(L_n \rightharpoonup L\) weakly in \(X^*\) \(\rightarrow\) \(\langle \tau(x), L_n \rangle \to \langle \tau(x), L \rangle\) \(\forall x \in X:\) namely
\[
    L_n x \to Lx \qquad \forall \; x \in X
\]
namely \(L_n\rightharpoonup^*\) weakly* in \(X^*\). In general the converse is false. However
\begin{proposition}
    If \(X\) is reflexive, then \(L_n \rightharpoonup L\) weakly in \(X^* \Leftrightarrow L_n \rightharpoonup^* L\) weakly* in \(X^*\)
\end{proposition}
\begin{proof}
    If \(X\) is reflexive, every element \(\Lambda\) of \(X^{**}\) is of type \(\Lambda = \tau(x)\) for some \(x\)
\end{proof}
\begin{proposition}
    \(X\) banach space, \(X^*\) dual space, \(L_n\rightharpoonup^* L\) in \(X^*\). Then
    \begin{itemize}
        \item The weak * limit is unique
        \item \(\{L_n\}\) is bounded
        \item \(\norm{L}_{X^*} \leq \liminf _{n \to \infty} \norm{L_n}_{X^*}\)
        \item If in addition \(x_n \to x\) strongly in \(X\) \(\Rightarrow\) \(L_nx_n \to Lx\) 
    \end{itemize}
\end{proposition}
\subsection{Banach-Alaoglu theorem}
\begin{theorem}[Banach Alaoglu]
    \(X \) separable Banach space. Then every bounded sequence in \(X^*\) has a weakly* convergent subsequence. (bounded sets in \(X^* \) sequentially compact for the weak* convergence)
\end{theorem}

\begin{proof}
    \(\{L_n\}\) bounded sequence in \(X^*\), namely
    \[
        \sup_{n} \norm{L_n}_{X^*} = M < \infty
    \]
    Since \(X\) is separable, \(\exists \; \{x_k\}_{k \in \natural}\) dense in \(X\). 
    Now, consider \(\{L_n x_1\}:\) it is bounded in \(\real\):
    \[
        \abs{L_n x_1} \leq \norm{L_n}_{X^*} \norm{x_1}_X \leq M \norm{x_1}_X < \infty
    \]
    \(\Rightarrow \exists \; \{L_{n_j}\}\) s.t. \(L_{n_j} x_1 \to l_j\) in \(\real\). Now, consider \(\{L_{n_j} x_2\}\): it is bounded,
    \[
        \abs{L_{n_j}x_2} \leq \norm{L_{n_j}}_{X^*} \norm{x_2}_X \leq M \norm{x_2}_X < \infty
    \]
    \(\Rightarrow \exists\; \{L_{n_{ij}}\}\) subsequence of \(\{L_{n_j}\}\) s.t. \( L_{n_{ij}} x_2 \to l_2\) in \(\real\)
    We can iterate the process. \(\forall \; k \quad \{L_n^k\}\) is a subsequence of \(\{L_n^{k-1}\}\). \(\Rightarrow \{L_n^k\}\) is a subsequence of \(\{L_n^j\}\) \(\forall i < k\). In particular, 
    \[
        L_n^k x_j \to l_j \quad \forall \; j \leq k
    \]
    We pick up \(T_n = L_n^n\) (diagonal selection). By construction, \(\forall m \in \natural\) fixed, \(\{T_n: n \geq m\}\) is a subsequence of \(\{L_n^m: n \geq m\}\) 
    \[
        \Rightarrow T_n x_m \to l_m \quad \text{ as } n \to \infty
    \]
    We want to show now that \(T_n x \to l_x\) \(\forall x \in X\), and that \(l_x = Tx\) is such that \(T \in X^*\). Since \(\{x_k\}\) is dense, \(\forall \; x \in X\) and \( \forall\; \epsilon >0\) \(\exists\; k \in \natural \) s.t. 
    \[
        \norm{x-x_k}_X < \frac{\epsilon}{2M}
    \]
    Thus
    \[
        \abs{T_n x - T_m x} \leq \abs{T_n x - T_n x_k} + \abs{T_n x_k - T_m x_k} + \abs{T_m x_k - T_m x} \leq 
    \]
    \[
    \leq \norm{T_n}_{X^*} \norm{x -x_k}_X + \abs{T_n x_k - T_m x_k} + \norm{T_m}_{X^*} \norm{x -x_k}_X \leq
    \]
    \[
        \leq M \frac{\epsilon}{2M} + \abs{T_n x_k - T_m x_k} + M \frac{\epsilon}{2M} 
        < \epsilon + \abs{T_n x_k - T_m x_k} < 2 \epsilon
    \]
    \(\forall \; n, m > \overline{n}\), since \(\{T_n x_k\}\) is convergent and so a Cauchy sequence.

    This means that \(\{T_n x\}\) is a Cauchy sequence in \(\real\)
    \[
        T_n x \to l_x \text{ in } \real \quad \forall x \in \real
    \]
    It only remains to show that \(l_x = Tx\) for some \(T \in X^*\). This is a consequence of a corollary of Banach Steinhaus.

    To sum up: \(\{L_n\}\) bounded in \(X^*\)
    \[
        \Rightarrow \exists \; \{T_n\} \text{ subsequence s.t. } T_n x \to Tx
    \]
    for every \(x \in X\), namely \(T_n\rightharpoonup^* T\) in \(X^*\)
\end{proof}

\begin{theorem}[Variant of BA for reflexive spaces]
    \(X\) reflexive and Banach. Then every bounded sequence in \(X\) has a weakly convergent subsequence
\end{theorem}

\begin{proof}
    For simplicity, we assume that \(X\) is separable (not necessary). \(X \) separable and reflexive \(\Rightarrow X^*\) is separable. \(\tau:X \to X^{**}\) canonical map: it is an isometric isometry.

    \(\{x_n\}\) bounded sequence in \(X \Leftrightarrow \{ \tau(x_n) \}\) is bounded in \(X^{**} = (X^*)^*\)
        
    \(\Rightarrow \) by Banach Alaoglu, \(\exists \; \{x_{n_k}\}\) s.t. \(\tau(x_{n_k})\rightharpoonup^* \Lambda \) in \(X^{**}\):
    \[
        {}_{X^{**}}\langle\tau(x_{n_k}), L \rangle_{X^*} \to {}_{X^{**}}\langle\Lambda, L \rangle_{X^*} \quad K \to \infty
    \]
    \(\forall\; L \in X^*.\) Since \(X\) is reflexive, \(\forall \; \Lambda \in X^{**} \) \(\exists ! \; x \in X\) s.t. \(\Lambda = \tau (x)\). Therefore, 
    \[
        L x_{n_k} = {}_{X^{**}}\langle\tau(x_{n_k}), L \rangle_{X^*} \to {}_{X^{**}}\langle \Lambda, L \rangle_{X^*} = Lx
    \]
    \(\forall\; L \in X^*\). We proved that 
    \[
        \lim_{k \to \infty} L x_{n_k} = Lx \qquad \forall\; L \in X^*
    \]
    namely \(x_{n_k} \rightharpoonup x\) in \(X\)
\end{proof}

