\section{Product measures space}
\subsection{Construction of product measure spaces}
\( (X, \mathcal{M}, \mu), (Y, \mathcal{N}, \nu) \) measure spaces.
The goal is to define a measure space on \(X \times Y\)
\begin{definition}
    We call \textbf{measurable rectangle} in \(X \times Y\) a set of type \(A \times B\) where \(A \in \mathcal{M}, B \in \mathcal{N}\)
    \[  R = \{ A \times B \subset X\times Y \text{ s.t. } A \in \mathcal{M}, B \in \mathcal{N}\}\]
    We define the product \(\sigma\)-algebra \(\mathcal{M} \otimes \mathcal{N}\) as \(\sigma_0(R)\). \\
    This is a \(\sigma\)-algebra in \(X \times Y\)
\end{definition}

\begin{definition}
    Let \(E \subset X \times Y \). For \( \bar{x} \in X \) and \(\bar{y} \in Y \) we define
\[ 
    \begin{array}{ll}
        E_{\bar{x}} = \{ y \in Y: \left( \bar{x}, y \right) \in E \} \subseteq Y & \qquad \bar{x} \text{-section of } E \\
        E_{\bar{y}} = \{ x \in X: \left( x, \bar{y} \right) \in E \} \subseteq X & \qquad \bar{y} \mbox{-section of } E \\
    \end{array}
\]
\end{definition}

\begin{proposition}
    \(\left( X, \mathcal{M} \right), \left( Y, \mathcal{N} \right)\) measurable spaces. \(E \in \mathcal{M} \otimes \mathcal{N}\) \\
    Then \(E_x \in \mathcal{M} \) and \(E_y \in \mathcal{N} \) 
    \(\Rightarrow \) we can define 
    \[
    \begin{array}{rlrl}
        \phi : & X \rightarrow \left[ 0, \infty \right] & \qquad  \psi : &Y \rightarrow \left[ 0, \infty \right] \\
                & x \mapsto \nu(E_x) & & y \mapsto \mu(E_y) 
        
    \end{array}    
    \]
\end{proposition}


\begin{theorem}
    If \(\left(X, \mathcal{M}, \mu \right)\) and \(\left(Y, \mathcal{N}, \nu \right)\) are \(\sigma\) finite spaces, then:
    \begin{enumerate}
        \item \(\phi\) is \(\mathcal{M}\)-measurable and \(\psi\) is \( \mathcal{N}\)-measurable
        \item we have that \(\int_X \nu(E_x) \, d\mu = \int_Y \mu(E_y) \, d\nu \)
    \end{enumerate}
\end{theorem}

Using the fact that \(\mu \) and \(\nu\) are measures, and that \(\int\) of non-negative function is a measure, we deduce the following

\begin{theorem}[Iterated integrals for characteristic functions]
    \(\mu \otimes \nu : \mathcal{M} \otimes \mathcal{N} \rightarrow \mathbb{R} \) defined by
    \[ \left(\mu \otimes \nu \right)(E) = \int_X \nu(E_x) \, d\mu = \int_Y \mu(E_y) \, d\nu\]
    is a measure, the product measure.
\end{theorem}

\begin{remark}[On the complection of product measure spaces] 
    \(\left(X, \mathcal{M}, \mu \right), \left(Y, \mathcal{N}, \nu \right)\) complete measures spaces. 
    
    In general, it is not true that \((X \times Y, \mathcal{M} \otimes \mathcal{N}, \mu \otimes \nu)\) is complete.
\end{remark}
\begin{example}
    \(X = Y = \real\), \(\mathcal{M} = \mathcal{N} = \mathcal{L}(\real), \mu = \nu = \lambda\).
    
\end{example}
Given \(A \mbox{ non meas. set }, A \subseteq [0,1], B = \left\{ y_0 \right\}, E = A \times B\). 
If \(E\) were measurable, then its sections must be measurable. But \(E_{y_0} = A\) which is not measurable.

However, \(E\) is negligible:
\[
    E \subseteq [0,1] \times \left\{ y_0 \right\}, \mbox{ and } \left(\lambda \otimes \lambda\right)\left([0,1] \times \left\{ y_0 \right\}\right) = 0
\]
Then \((\lambda \otimes \lambda)\) is not complete 
\[
    \Rightarrow (\real^2, \mathcal{L}(\real) \otimes \mathcal{L}(\real), \lambda \otimes \lambda) \neq (\real^2, \mathcal{L}(\real^2), \lambda_2)
\]
\begin{theorem}
    Let \(\lambda_n\) be the Lebesgue measure in \(\mathbb{R}^n\). 
    If \(n= K+m\), then \(\left(\mathbb{R}^n, \mathcal{L}(\mathbb{R}^n), \lambda_n \right)\) is the completion of \( ( \real^k \times \real^m, \mathcal{L}(\real^k) \otimes \mathcal{L}(\real^m),\lambda_k \otimes \lambda_m )\)
\end{theorem}
\subsection{Integration on product spaces}
\((X, \mathcal{M}, \mu), (Y, \mathcal{N},\nu)\) measure spaces. \(f : X \times Y \to \barreal\) measurable.

If \(f \geq 0\), then 
\[
    \iint_{X \times Y} f d\mu\otimes d\nu
\]
Goal: obtain a formula of iterated integral like the one in Analysis 2.

\(\forall \; \bar{x} \in X\) and \(\bar{y} \in Y\), we define
\[
    \begin{array}{llll}
        f_{\bar{x}} :& Y \to \barreal & f_{\bar{y}}:& X \to \barreal  \\
        & y \mapsto f(\bar{x}, y) & & x \mapsto f(x, \bar{y})

    \end{array}
\]
\begin{proposition}
    If \(f\) is measurable \(\Rightarrow\) \(f_{\bar{x}}\) is \((\mathcal{N}, \boreal)\)-measurable and \(f_{\bar{y}}\) is \((\mathcal{M}, \mathcal{B}(\barreal))\)-measurable.
    Then we can consider
    \[
        \begin{array}{ll}    
        \phi : X \to \barreal & 
        \phi(x) = \int_Y f_x d\nu = \int_Y f(x,y) \underbrace{d\nu(y)}_{dy} \\
        \psi : Y \to \barreal &
        \psi(y) = \int_X f_y d\mu = \int_X f(x,y) d\mu(x)
    \end{array}
    \]
\end{proposition}
\subsection{Tonelli's theorem}
\noindent\underline{Questions:} what is the solution of \(\iint_{X \times Y}\), \(\phi\) and \(\psi\)?

\begin{theorem}[Tonelli and Fubini's theorem]
    \((X, \mathcal{M}, \mu)\) and \((Y, \mathcal{N}, \nu)\) complete measure spaces and \(\sigma\)-finite. \\
    Suppose that \(f\) is \((\mathcal{M} \otimes \mathcal{N}, \mathcal{B}(\barreal))\)-measurable and that \(f > 0\) a.e. on \(X \times Y\). Then \(\psi\) and \(\phi\) are measurable and
    \[
        \iint_{X \times Y} f d\mu \otimes d\nu = \int_X \phi(x) \, d\mu(x) = \int_Y \psi(y) \, d\nu(y) \tag*{Integration formula}
    \]
    Equally holds also if one of the integrals is \(\infty\).
    \[
        \begin{array}{l}
            \int_X \phi(x) \, d\mu(x) = \int_X \left(\int_Y f(x, y) \, d\nu(y) \right) \, d\mu(x) \\
            \int_Y \psi(y) \, d\nu(y) = \int_Y \left(\int_X f(x, y) \, d\mu(x) \right) \, d\nu(y)      
    \end{array}  
    \]
\end{theorem}
\begin{remark}
    The double integral can be reduced to single integrals, iterated. Moreover, we can always change the order of the integrals.
    For sign changing functions the situation is more involved.
\end{remark}
\begin{theorem}[Fubini's theorem]
    \((X, \mathcal{M}, \mu)\) and \((Y, \mathcal{N}, \nu)\) complete measure spaces and \(\sigma\)-finite.
    If \(f \in L^1(X \times Y)\), then \(\psi\) and \(\phi\) defined above are measurable, the integration formula holds, and all the integrals are finite.
\end{theorem}
\noindent\underline{Question}: how to check if \(f\in L^1(X \times Y)\)? Typically, to check that \(f \in L^1(X \times Y)\) one uses Tonelli: 
\[
    f \in L^1(X \times Y) \Leftrightarrow \iint_{X \times Y} \abs{f} \, d\mu \otimes d\nu
\]
We use Tonelli to check that this is finite. 
If \(\iint_{X \times Y} \vert f \vert d\mu \otimes d\nu < \infty\) then we can apply Fubini for \(\iint_{X \times Y} f d\mu \otimes d\nu\)
\begin{remark}
    the proof of Fubini's and Tonelli's theorems is based for the iterated integrals for characteristic functions.
    Note that 
    \[(\mu \otimes \nu)(E) = \begin{array}{l}
        \int_X \phi(x) \, d\mu(x) = \int_X \left(\int_Y f(x, y) \, d\nu(y) \right) \, d\mu(x) \\
        \int_Y \psi(y) \, d\nu(y) = \int_Y \left(\int_X f(x, y) \, d\mu(x) \right) \, d\nu(y)
    \end{array}
    \]
\end{remark}
\begin{remark}
    Sometimes double integrals are very useful to compute single integrals.
\end{remark}
\begin{example}    
    \[\int_{-\infty}^{+\infty}e^{-x^2} = \sqrt{\pi}\]
\end{example}
