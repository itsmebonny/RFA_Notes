
\section{The Lebesgue integral}
\begin{notation}
    \[
        \begin{array}{c}
            \left\{ x \in X : f(x) \geq 0 \right\} = \left\{ f \geq 0 \right\} \\
            \left\{ x \in X : f(x) > 0 \right\} = \left\{ f > 0 \right\}   \\
            \vdots
        \end{array}
        \]
    \((X, \mathcal{M}, \mu)\) complete measure space.
    We consider measurable functions \(f: X \to [0, +\infty]\)

    \underline{Convention}: we define 
    \[
        \begin{array}{l}
            a + \infty = +\infty \quad \forall \; a \in \real \\
            a \cdot (+\infty) = \begin{cases}
                +\infty & \mbox{if } a \neq 0, a > 0 \\
                0 & \mbox{if } a = 0
            \end{cases}        
        \end{array}
    \]
    With this convention, \(+ \) and \( \cdot\) of measurable functions are measurable functions.
\end{notation}
\subsection{Integral of non-negative simple functions}
\begin{definition}
    Let \(s: X \to [0, +\infty]\) be a measurable simple function, 
    \[
        s(x) = \sum_{n=1}^m a_n \chi_{D_n}(\bar{x})
    \]
    where \(D_1,\ldots,D_m\) are measurable, disjoint, and \(\bigcup_{n=1}^m D_n = X\). Let also \(E \in \mathcal{M}\). Then we define 
    \[
        \int_E s \, d\mu := \sum_{n=1}^m a_n \mu(D_n \cap E)
    \]
\end{definition}
\begin{remark}
    Given a simple function \(s\):
    \[s:[a,b] \to \real, \lambda = \mu \Rightarrow \int_E s \, d\mu \mbox{ is the area under the curve}\]
\end{remark}
\begin{remark}
    There are several points:
    \begin{itemize}
        \item In the definition we have already used the convention \(\mu(D_n \cap E = +\infty) \quad \mbox{ for some }n\)
        \item \(E \in \mathcal{M} \Rightarrow \chi_E\) is a simple function
        \[
            \chi_E(x) = 1 \cdot \chi_E + 0 \cdot \chi_{X\setminus E}(x)
        \] 
        In this case 
        \[
            \int_X \chi_E \, d\mu = 1\cdot \mu(E) + 0 \cdot \mu(X\setminus E) = \mu(E)
        \]
        \item \(s\chi_E = \sum_{n=1}^m a_n\chi_{D_n \cap E} \Rightarrow \int_E s\, d\mu = \int_X s\chi_E \, d\mu\)
    \end{itemize}
\end{remark}
\subsection{Integral of non-negative measurable functions}
\begin{definition}
    \(f:X \to [0, +\infty]\) measurable, \(E \in \mathcal{M}\). The \textbf{Lebesgue integral} of \(f\) on \(E\), with respect to (w.r.t.) \(\mu\), is 
    \[
        \int_E f \, d\mu = \sup \left\{ \int_E s \, d\mu \; \bigg\lvert \begin{array}{l}s\text{ is simple} \\ 0 \leq s \leq f \end{array}\right\}
    \]
\end{definition}

\begin{enumerate}
    \item If \(f \) is simple, the definitions are consistent
    \item Also for \(f\) measurable: \( \int_E f \, d\mu = \int_X f \chi_E \, d\mu\)
    \item \( \left( \mathbb{N}, \mathcal{\mathbb{N}}, \mu_C \right)\). \(f: \mathbb{N} \to \mathbb{R}\) is a sequence \( \left\{ a_n \right\}_{n \in \mathbb{N}}\) \[ \int_\mathbb{N} \{a_n\} \, d\mu_C = \sum_{n=0}^\infty a_n\]
\end{enumerate}

\subsection{Basic properties of Lebesgue integral}

Let \(f, g : X \to \left[0, \infty\right]\) measurable. \(E, F \in \mathcal{M}, \ \alpha \geq 0\). Then: 
\begin{enumerate}
    \item \(\mu(E)=0 \Rightarrow \int_E f \, d\mu = 0\)
    \item \(f \leq g \) on \(E \Rightarrow \int_E f \, d\mu \leq \int_E g \, d\mu \)
    \item \(E \subset F \Rightarrow \int_E f \, d\mu \leq \int_F f \, d\mu\)
    \item \(\alpha \geq 0 \Rightarrow \int_E \alpha f \, d\mu = \alpha \int_E d \, d\mu\)
\end{enumerate}

\begin{remark}
    \(\left(\left[0, 1\right], \mathcal{L}(\left[0, 1\right]), \lambda  \right)\) \\
    Consider \(\chi_\mathbb{Q}\), it is the Dirichlet function on \(\left[0, 1\right]\). This is not Riemann integrable. \\ 
    However, \(\int_{\left[0,1\right]} \chi_{\mathbb{Q}} \, d\lambda = \lambda \left( \mathbb{Q} \cap \left[0,1\right] \right) =0 \)
\end{remark}
\subsection{Chebychev's inequality}

\begin{theorem}[Chebychev's inequality]
    \(f: X \to \left[0, \infty \right]\) measurable, \(c > 0\). Then \[ \mu\left(\{f \geq c \}\right) \leq \frac{1}{c} \int{\{f \geq c \}} f \, d\mu \leq \frac{1}{c} \int_X f \, d\mu \]
\end{theorem}
\begin{proof}
    \[ \int_X f \, d\mu \overset{X \supset \{f \geq c\}}{\geq} \int_{\{f \geq c \}} f \, d\mu \geq \int_{\{f \geq c\}} c \, d\mu 
    = c \int_{\{f \geq c\}} \, d\mu 
    = c \mu \left(\{f \geq c\}\right) \]
\end{proof}
\subsection{Measure defined by the integral}
\begin{theorem}
    \(s : X \to \left[0, \infty\right]\) simple. Define \(\phi : \mathcal{M} \to \left[0, \infty\right] \\ \phi(E) = \int_E s \, d\mu \\ \Rightarrow \phi \) is a measure. 
\end{theorem}
\begin{definition}[sigma additivity]
    \(\{E_n \subset \mathcal{M}\}\) disjoint, and let \(E = \bigcup_{n=1}^\infty E_n \Rightarrow s = \sum_{k=1}^m a_k \chi_{D_k} \; D_k \in \mathcal{M}
    \)
\end{definition}
\begin{proof}
    \(\mu(\emptyset) =0 \Rightarrow \phi(\emptyset)=0 \) by definition. 

    Then, by definition and since \(\mu\) is a measure and \(E \cap D_k = \bigcup_n (E_n \cap D_k)  \) 
    \[
        \phi(E) = \sum_{k=1}^m a_k \mu(D_k \cap E) = 
        \sum_{k=1}^\infty a_k \sum_{n=1}^\infty \mu(E_n \cap D_k)= 
    \]
    \[    
        \sum_{n=1}^\infty \left( \sum_{k=1}^m a_k \mu (E_n \cap D_k) \right) = 
        \sum_{n=1}^\infty \int_{E_n} s \, d\mu = 
        \sum_{n=1}^\infty \phi(E_n)
    \]
\end{proof}
\subsection{Vanishing Lemma}

\begin{theorem}[Vanishing Lemma]
    \(f: X \to \left[0, \infty\right]\) measurable. \(E \subset X \) measurable 
    \[\int_E f \, d\mu =0 \Leftrightarrow f=0 \text{ a.e. on } E \]
\end{theorem}
\begin{proof}
    \( \Leftarrow \) easy \\
    \( \Rightarrow \) Consider \( E \cap \{f >0\} = \bigcup_{n=1}^\infty \underbrace{\left(E \cap \{ f \geq \frac{1}{n} \} \right)}_{=:E_n} \) 
    
    Then \(\{E_n\}\) is an increasing sequence. By Chebychev 
    \[
        \mu (E_n) \leq \frac{1}{\frac{1}{n}} \int_E f \, d\mu =0 \quad \forall \; n \Rightarrow \mu(E_n)=0 \quad \forall \; n 
    \]
    \(\mu(E \cup \{f>0\}) \overset{\text{continuity}}{=} \lim_n \mu (E_n)=0\), namely \(f=0\) a.e. on \(E\)
\end{proof}

The \(\int\) does not see sets with 0 measure.

\begin{definition}
    If \( f:X \to \left[0, \infty\right] \) is measurable, and \( \int_X f \, d\mu < \infty \) then we say that \(f\) is integrable.
\end{definition}
\subsection{Monotone Convergence Theorem}
\begin{theorem}[Monotone Convergence - Beppo Levi]
    \(f_n:X\to \left[0, \infty\right]\) measurable. Suppose that 
    \begin{itemize}
        \item \(f_n(x) \leq f_{n+1}(x)\) for a.e. \(x \in X\) for every \(n\)
        \item \(f_n \to f \) a.e. on \(X\)
    \end{itemize} 
    Then \[ \int_X f \, d\mu = \lim_n \int_X f_n \, d\mu\]
\end{theorem}
\begin{proof}
    Part 1. \\
    Assume that the two hypothesis hold everywhere. First, if \(f\) is measurable 
    \[
        \int_X f_n \, d\mu \nearrow \quad \Rightarrow \exists \; \alpha = \lim_n \int_X f_n \, d\mu
    \]
    
    Also, \(f_n \leq f \) everywhere \(\Rightarrow \int_X f_n \, d\mu \leq \int_X f \, d\mu \quad \forall \; n\) 
    \[
        \Rightarrow \alpha \leq \int_X f \, d\mu 
    \]
    We want to show that also \(\geq \) is true. Let \(s\) be a simple function s.t. \(0 \leq s \leq f\) and \(c \in \left(0,1\right)\)
    Let \(E_n = \{f_n \geq cs\} \in \mathcal{M}\)
    \begin{itemize}
        \item \(E_n \subset E_{n+1} \; \forall \; n:\) 
        \\ if \(x \in E_n, \) then \(f_n(x) \geq cs(x) \Rightarrow f_{n+1}(x) \geq cs(x)\) \\ \(\Rightarrow f_{n+1}(x) \geq f_n(x) \geq cs(x) \Rightarrow x \in E_{n+1}\)
        \item Moreover, \(X = \bigcup_{n=1}^\infty E_n\). Indeed: 
        \\ - if \(f(x)=0\), then \(s(x)=0 \Rightarrow f_1(x)=0 = cs(x), \; x \in E_1\) 
        \\ - if \(f(x)>0\), then \(cs(x) < f(x)=\lim_n f_n(x)\) since \(s \leq f \) and \(c <1\) 
        \\ \(\Rightarrow cs(x) < f_n(x)\) for \(n \) sufficiently large, namely \(x \in E_n \) for \(n \) sufficiently large. 
    \end{itemize} 
    Therefore, 
    \[
        \alpha \geq \int_X f_n \, d\mu \geq \int_{E_n} f_n \, d\mu \geq c \int_{E_n} s \, d\mu = c \phi(E_n)
    \]
    \(\forall \; n, \ \forall \; 0 \leq s \leq f, \forall \; c \in \left[0, 1\right]\quad \phi(E_n) = \int_{E_n} s \, d\mu\). 
    \(\phi\) is a measure, and \(\{E_n\} \nearrow\) \\
    Therefore, taking the limit when \(n \to \infty\) by continuity 
    \[
        \alpha \geq \lim_n c \int_{E_n} s \, d\mu = c \int_X s \, d\mu \; \quad \forall c \in \left[0, 1\right]
    \]
    Take the limit when \(c \to 1^-: \ \alpha \geq \int_X s \, d\mu  \quad \forall \; 0 \leq s \leq f \) \\
    Take the sup over s: \(\alpha \geq \int_X f \, d\mu \).
    We proved both inequalities, so the thesis holds. \\
    Part 2. \\
    Note that \(\{x \in X: \text{assumptions of the theorem do not hold}\}\) is a set of zero measure. Take \(F. \ X = E \cup F \) since we have the assumption on \(E\) and \(\mu (F)=0\). \\ 
    Then, by the Vanishing Lemma, since \((f - f \chi_E)=0\) a.e. and \((f_n - f_n \chi_E)=0\) we have that 
    \[ 
        \int_X f \, d\mu = \int_E f \, d\mu = \lim_n \int_E f_n \, d\mu = \lim_n \int_X f_n \, d\mu 
    \]
\end{proof}

\begin{corollary}[Monotone convergence for series]
    \(f_n : X \to [0, +\infty]\) measurable, then 
    \[
        \int_X \left( \sum_{n=0}^{\infty} f_n\right) \, d\mu = \sum_{n=0}^{\infty} \int_X f_n \, d\mu
    \]
\end{corollary}

\begin{theorem}[Approximation with simple functions]
    Given \((X, \mathcal{M})\) measure space, \(f: X \to [0, +\infty]\) measurable, then \(\exists\) a sequence \(\left\{ s_n \right\}\) of simple functions s.t. 
    \[
        0 \leq s_1 \leq \ldots \leq s_n \leq \ldots \leq f \qquad \text{point wise } \forall \; x \in X
    \]
    and 
    \[
        s_n (x) \to f(x) \qquad \forall \; x \in X \text{as } n \to \infty 
    \]
    Moreover, if \(f\) is bounded, then \(s_n \to f\) uniformly on \(X\) as \(n \to \infty\).
\end{theorem}

\begin{remark}
    There is also
    \[
        \int_X f \, d\mu = \sup \left\{ \int_X s \, d\mu \, \bigg\vert \begin{array}{l}s\text{ is simple} \\ 0 \leq s \leq f \end{array}\right\}
    \]
\end{remark}

But let \(\left\{ s_n \right\}\) be the sequence given by the simple approximation theorem. By monotone convergence 
\[
    \int_X f \, d\mu = \lim_n \int_X s_n \, d\mu
\]
\begin{example}
    \(f, g : X \to [0, +\infty]\). Then 
    \[
        \int_X (f+g) \, d\mu = \int_X f \, d\mu + \int_X g \, d\mu
    \]
\end{example}
\subsection{Fatou's Lemma}
\begin{lemma}[Fatou's Lemma]
    Given \(f_n: X \to [0, +\infty]\) measurable \(\forall \, n\). Then 
    \[
        \int_X (\liminf_n f_n) \, d\mu \leq \liminf_n \int_X f_n \, d\mu
    \]
    In particular, if \(f_n \to f\) a.e. on \(X\), then
    \[
        \int_X f \, d\mu \leq \liminf_n \int_X f_n d\mu
    \]
\end{lemma}

\begin{proof}
    Given that \((\liminf_n f_n)(x) = \lim_n (\underbrace{\inf_{k \geq n} f_k(x)}_{= g_n (x)})\). Now, for every \(x \in X\), \(\left\{ g_n(x) \right\}\nearrow\)
    \[
        g_{n+1}(x) = \inf_{k \geq n+1} f_k(x) \geq \inf_{k \geq n} f_k(x) = g_n (x)
    \]
    Also, \(g_n \geq 0\) on \(X\). Thus, by monotone convergence
    \[
        \int_X \liminf_n f_n \, d\mu = \int_X \lim_n g_n \, d\mu = \lim_n \int_X g_n \, d\mu = \liminf \int_X g_n \, d\mu
    \]
    Now, note that \[g_n (x) = \inf_{k\geq n} f_k(x) \leq f_n(x) \leq \liminf_n \int_X f_n \, d\mu\]
\end{proof}
\subsection{Integration of series of non-negative functions}
\begin{theorem}[\(\sigma\)-additivity of \(\int\)]
    Given \((X, \mathcal{M}, \mu)\) measure space, \(\oldphi : X \to [0, +\infty]\). Define \(\nu(E) = \int_E \oldphi \, d\mu\), with \(E \in \mathcal{M}\). 
    \(\nu : \mathcal{M} \to [0, +\infty]\) is a measure. Moreover, let \(f:X \to [0, +\infty]\) measurable
    \[
        \int_X f \, d\nu = \int_X f\oldphi \, d\mu \tag*{*}
    \]
\end{theorem}
\begin{proof}
    \noindent\underline{\(\nu\) is a measure}:  

    \(\nu(\emptyset) = 0\), since \(\mu(\emptyset) = 0\). 
    Now, let \(E = \bigcup_{k=1}^{\infty} E_k\), \(\left\{ E_k \right\}\) disjoint. Then 
    \[
        \nu(E) = \int_X \oldphi\chi_{E} \, d\mu = \int_X \oldphi\sum_n \chi_{E_n} \, d\mu \underset{\substack{\text{\scriptsize{monot. conv.}} \\ 
            \text{\scriptsize{for }}\textstyle\sum}}{=}  \sum_n \int_X \oldphi \chi_{E_n} \, d\mu = \sum_n \int_{E_n} \oldphi \, d\mu = \sum_n \nu(E_n) 
    \]
    We have proven \(\sigma\) additivity, so \(\nu\) is a measure.

    \noindent\underline{(*) holds}:
    Let \(E \in \mathcal{M}\). Then
    \[
        \int_X \chi_E \, d\nu = \int_E 1 \, d\nu = \nu(E)  = \int_E \oldphi \, d\mu = \int_X \oldphi \chi_E \, d\mu
    \]
    This shows that \((*)\) holds for \(\chi_E\), \(\forall \; E\). Then it holds for simple functions. \\
    Let now f be any measurable function, positive. Then we can take \(\{s_n\}\) given by the simple approximation theorem
    \[
        \int_X f \, d\nu \overset{\text{monot}}{=} \lim_n \int_X s_n \, d\nu = \lim_n \int_X s_n \oldphi \, d\mu \overset{\text{monot}}{=} \int_X f \oldphi \, d\mu 
    \]
    which is \((*)\) 
\end{proof}

\begin{remark}
    \(X, \mathcal{M}, \mu\) complete measure space. Then, by the vanishing lemma, it is not difficult to deduce that 
    \[
        f=g \text{ a.e. on } X \Leftrightarrow \int_E f \, d\mu = \int_E g \, d\mu \qquad \forall\; E \in \mathcal{M}
    \]
    The \(\int\) does not see differences of sets with 0 measure. As a consequence, in all the theorems, it is sufficient to assume that the assumptions are satisfied a.e. 
\end{remark}


\subsection{Integrable functions}
\(X, \mathcal{M}, \mu\) complete measure space. \\
\(f: X \to \barreal = [-\infty, \infty]\) measurable. Recall \(f= f^+ - f^- \) where \(f^+ = \max{\{f, 0\}}\), \(f^- = -\min{\{f, 0\}} \) and \(|f|= f^+ + f^-\). 
Note that both are positive and measurable. 

\begin{definition}
    We say that \(f:X \to \barreal\) measurable is integrable on \(X\) if 
    \[
        \int_X |f| \, d\mu < \infty
    \]
\end{definition}

If \(f\) is integrable, we define \(\int_X f \, d\mu = \int_X f^+ \, d\mu + \int_X f^- \, d\mu\) 
\subsection{The set \texorpdfstring{\(\mathcal{L}^1\)}{L1}}
The set of integrable functions is denoted by 
\[
    \mathcal{L}^1 (X, \mathcal{M}, \mu) := \{f:X \to \barreal \text{ integrable} \} 
\]
\[
    \mathcal{L}^1 (X, \mathcal{M}, \mu) 
    = \mathcal{L}^1 (X) 
    = \mathcal{L}^1 
\]

If \(E \in \mathcal{M}\), we define
\[
    \int_E f \, d\mu = \int_X f \chi_E \, d\mu
\]

\begin{remark}
    \(f \in \mathcal{L}^1(X) \Rightarrow \int_X f \, d\mu \in \real\). Indeed, \(0 \leq f^\pm \leq |f|\)
    \[
        \Rightarrow 0 \leq \int_X f^+ \, d\mu ,\ \int_X f^- \, d\mu \leq \int_X |f| \, d\mu < \infty 
    \] 
    \[
        \Rightarrow \int_X f \, d\mu = \int_X f^+ \, d\mu - \int_X f^- \, d\mu \in \real
    \]
\end{remark}
\subsection{Triangle inequality}
\begin{proposition}
    \(f : X \to \barreal \) measurable. Then
    \begin{enumerate}
        \item \(f \in \mathcal{L}^1 \Leftrightarrow |f| \in \mathcal{L}^1 \Leftrightarrow \) both \(f^+, \ f^-\) \(\in \mathcal{L}^1\)
        \item \(f \in \mathcal{L}^1 \), then 
        \[
            \left| \int_X f \, d\mu \right| \leq \int_X |f| \, d\mu  \tag{triangle inequality}
        \]
    \end{enumerate}
\end{proposition}

\begin{proof}
    Of the second part.
    \[
        \left| \int_X f \, d\mu \right| = \left| \int_X f^+ \, d\mu + \int_X f^- \, d\mu\right| \leq \int_X f^+ \, d\mu + \int_X f^- \, d\mu = \int_X |f| \, d\mu  
    \]
\end{proof}
\subsection{\texorpdfstring{\(\mathcal{L}^1\)}{L1} is a vector space}
\begin{proposition}
    \(\mathcal{L}^1(X, \mathcal{M}, \mu)\) is a vector space, and \(f, \, g \in \mathcal{L}^1\), \(\alpha \in \real\)
    \[
        \Rightarrow \int_X \left(\alpha f + g \right) \, d\mu = \alpha \int_X f \, d\mu + \int_X g \, d\mu 
    \]  
    by linearity of the integrals.
\end{proposition}

Many results can be extended from non-negative functions to general functions.

\begin{theorem}
    \((X, \mathcal{M}, \mu)\) complete measure space. \(f\), \(g \in \mathcal{L}^1\). Then
    \[
        f= g \text{ a.e. on } X \Leftrightarrow \int_X |f-g| \, d\mu =0 \Leftrightarrow \int_E f \, d\mu = \int_E g \, d\mu \qquad \forall \; E \in \mathcal{M} 
    \]
\end{theorem}
\subsection{Dominated convergence theorem}
The most relevant theorem for convergence is the following
\begin{theorem}[Dominated convergence theorem]
    \(\{f_n\}\) sequence of measurable functions \(X \to \barreal\). Suppose that
    \begin{enumerate}
        \item \(f_n \to f \) a.e. on \(X\)
        \item \(\exists \; g : X \to \barreal \), \(g \in \mathcal{L}^1(X)\), such that \(|f_n(x)| \leq g(x)\) a.e. on \(X\) \(\forall \; n \in \mathbb{N}\)
    \end{enumerate}
    Then \(f \in \mathcal{L}^1\) and 
    \[
        \lim_n \int_X |f_n -f| \, d\mu = 0 
        \qquad \left( \Rightarrow \int_X f \, d\mu = \lim_n \int_X f_n \, d\mu \right)  
    \]
\end{theorem}
\begin{proof}
    Note that \(f_n \in \mathcal{L}^1\) \(\forall \; n\), since \(|f_n| \leq g\) and we have the monotonicity of \(\int\) for non-negative functions
    \[
        |f_n(x)| \leq g(x) \quad
        n \to \infty \qquad
        |f(x)| \leq g(x) \text{  a.e. on } X 
    \]
    \[ 
        \Rightarrow f \in \mathcal{L}^1(X)
    \]  
    Now, consider \(\oldphi_n = 2g - |f_n - f|\). We have
    \[
        |f_n-f| \leq |f_n|+|f| \leq 2g \quad \text{ a.e. on }X \quad \oldphi_n \geq 0 \quad \text{ a.e. on }X  
    \]
    We can use Fatou's lemma:
    \[\begin{array}{l}
        \int_X (\underbrace{\liminf_n \oldphi_n}_{\stackbelowlittle{\mbox{= 2g a.e.}}{\int_X 2g \; d\mu}}) \; d\mu \leq \liminf_n \int_X \oldphi_n \, d\mu = \liminf_n \int_X (2g - \abs{f_n - f}) \, d\mu = \\
        = \int_X 2g \, d\mu + \liminf_n (-\int_X \abs{f_n - f} \, d\mu) =\int_X 2g \, d\mu - \limsup_n \int_X \abs{f_n - f} \, d\mu
    \end{array}
    \]
    Subtracting \(\int_X 2g \, d\mu\) from both sides 
    \[
        0 \leq -\limsup_n \int_X \abs{f_n - f} \, d\mu \Rightarrow 0 \leq \liminf_n \int_X \abs{f_n - f} \, d\mu \leq \limsup_n \int_X \abs{f_n - f} \, d\mu \leq 0 
    \]  
\end{proof}
\begin{remark}
    If \(\mu(X) < +\infty\), and \(\exists \; M > 0\) s.t. \(\abs{f_n} \leq M\) a.e. on \(X, \; \forall \; n\), then we can take \(g = M\) as dominating function.
\end{remark}

\subsection{Comparison between Riemann and Lebesgue integrals}

Let \(f : I \subset \real \to \real\), \(I\) interval, be bounded. Assume also that \(I\) is closed and bounded.
\begin{theorem}
    Let \(f\) be Riemann-integrable on \(I \ (f \in R(I))\). Then 
    \[
        f\in \mathcal{L}^1(I, \mathcal{L}(I), \lambda)
    \]
    and 
    \[
        \int_I f \, d\lambda = \int_I f(x) \, dx
    \]
\end{theorem}
\begin{theorem}
    \(f \in R(I) \Leftrightarrow f\) is continuous on \(x\), for a.e. \(x \in I\).
\end{theorem}
\begin{example}
    \(\chi_{\mathbb{Q}}\) on \([0,1]\) is not Riemann integrable, because it is discontinuous at any point. Note that, instead, \(\chi_{\mathbb{Q}} = 0\) a.e. on \([0,1]\) \(\Rightarrow \int_{[0,1]} \chi_{\mathbb{Q}} \, d\lambda = 0\).
\end{example}
Let \(f \not \in R(I)\). Is it true that \(\exists \; g \in R(I)\) s.t. \(g = f\) a.e. on \(I\)? No.

For instance, consider \(T_{\mathcal{E}}\), the generalized Cantor set (\(\lambda(T_{\mathcal{E}}) = 0\)) and then consider \(\chi_{T_{\mathcal{E}}}\). \\
In general, \(\chi_{A}\) is discontinuous on \(\delta A\).  But \(T_{\mathcal{E}}\) has no interior parts \(\Rightarrow T_{\mathcal{E}} = \delta T_{\mathcal{E}}\) \(\Rightarrow \chi_{T_{\mathcal{E}}}\) is discontinuous on \(T_{\mathcal{E}}\), which has positive measure
\(\Rightarrow \) by the last theorem, \(\chi_{T_\epsilon}\) is not \(R(I)\)

Clearly 
\[
    \int_{[0,1]} \chi_{T_{\mathcal{E}}} d\lambda = \lambda(T_{\mathcal{E}})
\]
so \(\chi_{T_{\mathcal{E}}} \in \mathcal{L}^1([0,1])\).  

If \(g = \chi_{T_{\mathcal{E}}}\) a.e., then \(g\) is discontinuous at almost every part of \(T_{\mathcal{E}} \Rightarrow\) \(g\) is discontinuous on a set of positive measure \(\Rightarrow g \not \in R(I)\). 
So, the Lebesgue integral is a true extension of the Riemann one.

Regarding generalized integrals we have

\begin{theorem}
    \(-\infty \leq a < b \leq +\infty, \quad f \in R^g([a,b])\) where 
    \[
        R^g([a,b]) = \left\lbrace \mbox{Riemann-int functions on }[a,b]\mbox{ in the generalized sense} \right\rbrace
    \]
    Then, \(f\) is \(([a,b], \mathcal{L}([a,b]))\)-measurable. Moreover,
    \begin{enumerate}
        \item \(f \geq 0\) on \([a,b] \Rightarrow f \in \mathcal{L}^1([a,b])\)
        \item \(\vert f \vert \in R^g([a,b]) \Rightarrow f \in \mathcal{L}^1 ([a,b])\)
    \end{enumerate}
    and in both cases
    \[
        \int_{[a,b]} fd\lambda = \int_a^b f(x)dx
    \]
    If \(f\) is in \(R^g([a,b])\), but \(\vert f\vert \not \in R^g([a,b])\), then the two notions of \(\int\) are not really related
\end{theorem}
\begin{example}
    \(f(x) = \frac{\sin x}{x},  x \in [1, \infty]\)
    \[
        \int_1^{\infty} \vert f(x) \vert dx = +\infty \Rightarrow f \not \in \mathcal{L}^1([1, +\infty])
    \]
    But on the other hand
    \[
        \int_1^{\infty} \frac{\sin x}{x} dx = \lim_{\omega \to \infty} \int_1^{\omega} \frac{\sin x}{x} dx = \frac{\pi}{2}
    \]
\end{example}
\begin{proposition}
    \((X, \mathcal{M}, \mu)\) complete measure space. Let \(\{f_n\} \subseteq \mathcal{L}^1(X, \mathcal{M}, \mu)\). 
    
    Suppose that \(\sum_{n=1}^\infty \int_X |f_n| \, d\mu < \infty\)
    Then the series \(\sum_{n=1}^\infty f_n\) converges a.e. on \(X\), it is in \(\mathcal{L}^1(X)\) and 
    \[
        \int_X \left( \sum_{n=1}^\infty f_n  \right) \, d\mu = \sum_{n=1}^\infty \int_X f_n \, d\mu
    \]
\end{proposition}


\subsection{The spaces \texorpdfstring{\(\mathcal{L}^1\)}{L1} and \texorpdfstring{\(\mathcal{L}^\infty\)}{Linf}}
\((X, \mathcal{M}, \mu)\) complete measure space.
\[
    \mathcal{L}^1 = \left\lbrace f: X \to \barreal : \mbox{ f is integrable}\right\rbrace
\]
\(\mathcal{L}^1\) is a vector space. On \(\mathcal{L}^1\) we can introduce \(d : \mathcal{L}^1 \times \mathcal{L}^1 \to [0, +\infty)\) defined by 
\[
    d_1 (f,g) =\int_{X} \vert f-g \vert 
\]

It is immediate to check that 
\[
    d_1 (f, g) = d_1(g, f) \tag*{(symmetry)}
\]  

\[
    d_1(f, g) \leq d_1(f, h) + d_1(h, g) \; \;\forall f, g, h \in \mathcal{L}^1(X) \tag*{(triangular inequality)}
\]  
However, \(d_1\) is not a distance on \(\mathcal{L}^1(X)\), since 
\[
    d_1(f,g) = 0 \Rightarrow f=g \quad \mbox{a.e on }X
\tag*{(pseudo-distance)}\]
To overcome this problem, we introduce an equivalent relation in \(\mathcal{L}^1(X)\): we say that 
\[
    f \sim g \Leftrightarrow f = g \quad \mbox{a.e. on }X
\]
If \(f \in \mathcal{L}^1(X)\), we can consider the equivalence class
\[
    [f] = \left\lbrace g \in \mathcal{L}^1(X) : g = f \mbox{ a.e on }X \right\rbrace
\]
We define
\[
    L^1(X) = \frac{\mathcal{L}^1(X)}{\sim} = \{[f]: f \in \mathcal{L}^1(X)\}
\]
\(L^1(X)\) is a vector space, and on \(L^1(X)\) the function \(d_1\) is a distance: 
\[
    d_1([f], [g]) = 0 \Leftrightarrow \int_X |[f]-[g]|\, d\mu =0 \Leftrightarrow [f]= [g] \text{ a.e. } \Leftrightarrow f=g \text{ a.e. }
\]
To simplify the notations, the elements of \(L^1(X)\) are called functions, and one writes \(f \in L^1(X)\). With this, we mean that we choose a representative in \([f]\), and f denotes both the representative and the equivalence class. The representative can be arbitrarily modified on any set with \(0\) measure.

Another relevant space of measurable functions is the space of \textbf{essentially bounded} functions.
\begin{definition}
    \(f : X \to \overline{\mathbb{R}}\) measurable is called essentially bounded if \(\exists \; M > 0\) s.t.
    \[
        \mu(\left\lbrace x \in X : \vert f(x) \vert \geq M \right\rbrace) = 0
    \]
\end{definition}
\begin{example}
    \[f(x) = \begin{cases}
        1 & x > 0 \\
    +\infty & x = 0 \\
    0 & x < 0 \\
    \end{cases}
    \]
\end{example} 
For \(M > 1\), \(\lambda(\left\lbrace x \in \mathbb{R} : \vert f(x) \vert > M\right\rbrace) = \lambda(\left\lbrace 0 \right\rbrace) = 0 \Rightarrow f\mbox{ is essentially bounded}\).


If \(f\) is essentially bounded, it is well-defined the \textbf{essential supremum} of \(f\).
\[
    \underset{X}{\esssup} f := \inf \left\lbrace M > 0 \mbox{ s.t. } f \leq M \mbox{ a.e. on }X\right\rbrace = \inf \left\lbrace M > 0 \mbox{ s.t. } \mu(\{f \geq M \})=0 \right\rbrace
\]
It can also be defined an essential inf.
\begin{remark}
    Note that, by def of inf, \(\forall \; \epsilon > 0\) we have 
    \[
        f \leq (\underset{X}{\esssup} f) + \epsilon \qquad \text{a.e. on }X
    \]
\end{remark}
We define 
\[
    L^{\infty} (X, \mathcal{M}, \mu) = \frac{\mathcal{L}^{\infty}(X, \mathcal{M}, \mu)}{\sim}
\]
\(L^{\infty}(X)\) is a vector space, and it is also a metric space for \(d_{\infty}(f,g) = \underset{X}{\esssup} \vert f-g \vert\)